\documentclass[11pt]{article}

\usepackage{amsmath, amssymb, amsthm}
\usepackage{geometry}
\geometry{margin=1in}

\title{Dyadic Band Geometry and Carry Isolation\\
in Affine--Valuation Dynamical Systems}

\author{Author Name}
\date{\today}

\newtheorem{theorem}{Theorem}[section]
\newtheorem{lemma}[theorem]{Lemma}
\newtheorem{corollary}[theorem]{Corollary}
\newtheorem{definition}[theorem]{Definition}
\newtheorem{remark}[theorem]{Remark}

\begin{document}

\maketitle

\begin{abstract}
We introduce a dyadic coordinate framework for analyzing affine--valuation dynamical systems of the form
\[
T_{a,b}(n) = \frac{an+b}{2^{v_2(an+b)}},
\]
defined on odd integers. The framework decomposes dynamics into vertical scale motion (band transitions), horizontal position within dyadic bands, and valuation-driven collapse. We prove exact band-transition and logarithmic drift identities and formalize a sharp carry-buffer isolation theorem quantifying when lower binary blocks influence higher blocks under affine multiplication. The results apply uniformly to the family of maps $T_{a,b}$, including but not limited to the classical odd-only $3n+1$ map. No convergence claims are made.
\end{abstract}

\section{Introduction}

Affine maps followed by 2-adic normalization arise naturally in number-theoretic dynamics. Given fixed integers $a \ge 3$ (odd) and $b$, define
\[
T_{a,b}(n) = \frac{an+b}{2^{v_2(an+b)}}
\]
for odd integers $n$.

Such maps combine:
\begin{enumerate}
\item multiplicative expansion,
\item additive perturbation,
\item 2-adic valuation discharge.
\end{enumerate}

We introduce a dyadic geometric coordinate system separating these mechanisms and formalize binary carry-locality phenomena governing structural interactions across bit scales.

\section{Dyadic Band Coordinates}

\begin{definition}[Band index]
For $n \in \mathbb{N}$,
\[
b(n) = \lfloor \log_2 n \rfloor.
\]
Then
\[
2^{b(n)} \le n < 2^{b(n)+1}.
\]
\end{definition}

\begin{definition}[Band scale]
\[
B(n) = 2^{b(n)+1}.
\]
\end{definition}

\begin{definition}[Normalized band coordinate]
\[
x(n) = \frac{n}{B(n)} \in \left[\frac{1}{2},1\right).
\]
The mapping
\[
n \longleftrightarrow (b(n), x(n))
\]
is bijective.
\end{definition}

\begin{definition}[Remainder coordinate]
\[
r(n) = n - 2^{b(n)}, \quad 0 \le r(n) < 2^{b(n)},
\]
\[
R(n) = \frac{r(n)}{B(n)} \in \left[0,\frac{1}{2}\right).
\]
Then
\[
x(n) = \frac{1}{2} + R(n).
\]
\end{definition}

\section{Affine--Valuation Dynamics}

Let $a \ge 3$ be odd and $b \in \mathbb{Z}$. Define
\[
T_{a,b}(n) = \frac{an+b}{2^{v_2(an+b)}}
\]
for odd $n$.

\section{Vertical Motion}

\begin{lemma}[Exact band identity]
For odd $n$,
\[
b(T_{a,b}(n))
=
b(an+b)
-
v_2(an+b).
\]
\end{lemma}

\begin{proof}
\[
T_{a,b}(n) = \frac{an+b}{2^{v_2(an+b)}}.
\]
Taking base-2 logarithms and floors yields the identity.
\end{proof}

\begin{corollary}[Band displacement]
\[
\Delta b(n)
=
b(T_{a,b}(n)) - b(n)
=
b(an+b) - v_2(an+b) - b(n).
\]
\end{corollary}

\section{Logarithmic Drift Identity}

\begin{lemma}[Exact drift decomposition]
For odd $n$,
\[
\log_2 T_{a,b}(n)
=
\log_2 n
+
\log_2 a
-
v_2(an+b)
+
\epsilon(n),
\]
where
\[
\epsilon(n)
=
\log_2\!\left(1+\frac{b}{an}\right),
\]
and $|\epsilon(n)| = O(n^{-1})$.
\end{lemma}

\begin{proof}
\[
T_{a,b}(n)
=
\frac{an+b}{2^{v_2(an+b)}}
=
\frac{an}{2^{v_2(an+b)}}\left(1+\frac{b}{an}\right).
\]
Taking logarithms yields the result.
\end{proof}

\section{Carry-Buffer Isolation}

\subsection{Binary block decomposition}

Fix integers $m \ge 1$ and $t \ge 1$. Write
\[
n = H\,2^{m+t} + S,
\quad 0 \le S < 2^m.
\]

In binary, this corresponds to:
\begin{itemize}
\item lowest $m$ bits: tail $S$,
\item next $t$ bits: zeros (buffer),
\item higher bits: head $H$.
\end{itemize}

\begin{theorem}[Sharp carry-buffer criterion]
Let $a \ge 1$, $b \in \mathbb{Z}$, and define $y = an+b$. Then
\[
y = (aH)2^{m+t} + (aS+b).
\]
Define
\[
c = \left\lfloor \frac{aS+b}{2^{m+t}} \right\rfloor.
\]
Then
\[
\left\lfloor \frac{y}{2^{m+t}} \right\rfloor = aH + c.
\]
In particular,
\[
\text{Head unaffected by tail}
\iff
0 \le aS+b < 2^{m+t}.
\]
\end{theorem}

\begin{proof}
\[
y = a(H2^{m+t}+S)+b = (aH)2^{m+t} + (aS+b).
\]
Divide by $2^{m+t}$ and take floors:
\[
\left\lfloor \frac{y}{2^{m+t}} \right\rfloor
=
aH + \left\lfloor \frac{aS+b}{2^{m+t}} \right\rfloor.
\]
\end{proof}

\begin{theorem}[Prefix invariance]
If
\[
0 \le aS+b < 2^{m+t},
\]
then all bits of $y$ at positions $\ge m+t$ depend only on $H$, not on $S$.
\end{theorem}

\begin{proof}
Under the stated inequality,
\[
y = (aH)2^{m+t} + (aS+b),
\]
with $aS+b < 2^{m+t}$. Thus bits above position $m+t$ coincide with those of $aH$.
\end{proof}

\begin{corollary}[Uniform isolation condition]
If $b \ge 0$ and
\[
a(2^m - 1) + b < 2^{m+t},
\]
then isolation holds for every $S \in [0,2^m)$.
\end{corollary}

\section{Structural Decomposition}

The affine--valuation map separates into:
\begin{enumerate}
\item multiplicative expansion (controlled by $a$),
\item binary carry interaction (controlled by buffer size),
\item valuation discharge (controlled by $v_2(an+b)$).
\end{enumerate}

The dyadic band coordinate system isolates scale from intra-band structure and makes valuation discharge the sole driver of vertical collapse.

\section{Scope}

No convergence claims are made for any specific instance of $T_{a,b}$. The results provide structural tools applicable uniformly across affine--valuation dynamical systems.

\end{document}
